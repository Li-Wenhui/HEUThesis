% !TEX TS-program = XeLaTeX
% !TEX encoding = UTF-8 Unicode

%%%%%%%%%%%%%%%%%%%%%%%%%%%%%%%%%%%%%%%%%%%%%%%%%%%%%%%%%%%%%%%%%%%%%%
%
%  哈尔滨工程大学学位论文 XeLaTeX 模版 —— 其他附加文件 app02.tex
%
%  版本:1.0.0
%  最后更新:
%  修改者:Leo LiWenhui lwh@hrbeu.edu.cn
%  修订者:
%  编译环境1:Ubuntu 12.04 + TeXLive 2013/2014
%  编译环境2:Windows 7/8  + TeXLive 2013/2014
%
%%%%%%%%%%%%%%%%%%%%%%%%%%%%%%%%%%%%%%%%%%%%%%%%%%%%%%%%%%%%%%%%%%%%%

\appendix{附录~B~~哈尔滨工程大学学位论文撰写规范}

 \hei 说明:
 \kai
 这些规定偶有变动。

 请仔细查阅当年拿到的《哈尔滨工程大学研究生学位论文规范》。

 若发现不一致的地方, 请与我联系(lwh@vip.163.com)。

\vspace*{0.5cm}

\song
学位论文是表明作者从事科学研究取得创造性结果或有了新的见解,并以此为内容撰写而成的学术论文。研究生学位论文展示了研究生在科学研究工作中取得的成果并全面反映了研究生对本学科基础理论和专门知识的掌握程度,是申请和授予相应学位的基本依据。学位论文撰写是研究生培养过程的基本训练之一,必须按照确定的规范认真执行。

本论文规范按照《科学技术报告、学位论文和学术论文的编写格式》(GB 7713-87)、《文后参考文献著录规则》(GB 7714-87)以及《标准化工作导则标准编写的基本规定》(GB/T1.1-1993)制定。

本论文规范适用于我校博士、硕士研究生和本科生。学位论文除在字数、理论研究的深度及创造性成果等方面的要求不同以及特殊说明外,对其撰写规范的要求基本一致。


{\hei 一、论文装订格式的排列顺序\footnote{硕士研究生和本科生的要求与博士生有不同。}}

(一)封面

(二)论文英文题目

(三)原创性及授权声明

(四)学位论文使用授权书

(五)博士生自认为的论文创新点

(六)目录

(七)中文摘要

(八)英文摘要

(九)引言

(十)正文

(十一)中外文参考文献

(十二)攻博期间发表的与学位论文相关的科研成果目录

(十三)后记/致谢

{\hei 二、论文印制规格及要求}

(一)论文用A4张(210×297mm)标准大小的白纸打印;

(二)\colorbox{yellow}{正文部分研究生论文要求双面印制,本科生论文单面印制};

(三)版心尺寸160×241mm,版心页边距上、下设置为28mm,左、右设置为25mm,页眉、页脚设置为20mm。装订时上、下、右各切除3mm,学位论文成品版面大小为207mm*291mm。

{\hei 三、论文封面格式}

(一)分类号:必须在封面左上角注明分类号,一般应注明《中国图书资料分类法》的类号,
同时尽可能注明《国际十进分类法UDC》的类号.

(二)密级:论文必须按国家规定的保密条例在右上角注明密级(如系公开型论文则可不注明密级)。

(三)论文题目:题目必须用楷体标准一号字标注于明显的位置,应是集中概括论文最重要的内容,
一般不超过20个字,以有助于选定关键词和编制题录。题目不能用缩略词,首字母缩写字、字符、
代号和公式等,题目语意未尽,可用副标题补充说明。外语专业的论文题目一般采用英文,英文题目不宜超过10个实词。

(四)论文作者姓名:

(五)论文指导教师姓名:指导教师姓名必须是填写当年被学校批准招收博士生的教师。

(六)专业名称:专业名称必须是我校已有学位授予权的学科专业,并按国家颁布的学科专业目录中一级学科或二级学科名称印制。

(七)书脊(专指博士学位论文):书脊上应用仿宋体四号字于上方标明论文题目,下方注明研究生姓名。

(八)论文封面:统一用120克铜版纸,封面底色为白色。

{\hei 四、论文英文题目}

论文英文题目专用一页纸,“英文题目”用宋体二号字,其下“研究生姓名”用宋体四号字;外语专业应为中文题目。

{\hei 五、原创性及授权声明}

“原创性及授权声明”用黑体小二号字,内容用宋体四号字。

为了加强学风、学术道德建设,规范学术行为,提高学位论文质量,确保学位授予的权威性、严肃性,学校对学位论文撰写特别强调以下几点:

(一)凡申请学位人员须对自己的学位论文负责,在提交的学位论文的英文题目后页(中文摘要前页)增设一页书面声明,即“郑重声明”。

(二)学位论文中的引证、引述处须注明出处。

(三)学位论文后所附参考文献,必须是申请学位人员真正阅读和参考过的文献。

(四)合作科研及成果,应在致谢或后记中有相关说明,避免产权纠纷。

(五)学位论文中没有原创性及授权声明的,不能参加论文答辩。

{\hei 六、中文摘要}

“中文摘要”用黑体小二号字,内容用宋体小四号字,页码用罗马数字单独编排,并标注在每页页脚中部。

{\hei 七、英文摘要}

“英文摘要”用加粗Times New Roman小二号字,内容用Times New Roman小四号字,页码续接中文摘要的页码。

{\hei 八、论文关键词}

每篇论文必须选取3--5个以上中、英文关键词,排在其论文摘要的左下方,用黑体小四号字。

{\hei 九、目录}

目录是论文的提纲,也是论文组成部分的小标题.
排列顺序是:1、中文摘要 2、英文摘要 3、引言 4、正文章节 5、中外文参考文献 6、攻博期间发表的科研成果目录
7、后记(可不要此项)。 并对每项标明页码。

{\hei 十、引言(绪论)}

论文的页码由引言(绪论)的首页开始,作为第1页,并为右页,一律用阿拉伯数字连续编排页码,必须统一标注在每页页脚中部。

{\hei 十一、正文}

正文是学位论文的核心部分,必须由另页开始,一级标题之间换页,二级标题之间空行;内容一律用宋体小四号字,
字间距设置为标准字间距,行间距设置为最小值20磅,各章、节应有序号。

{\hei 十二、参考文献}

参考文献用黑体四号字,内容用宋体五号字。

{\hei 十三、学位论文印刷份数}

由培养单位根据需要决定印刷份数. 在学位论文定稿后,可先印刷数份给论文评阅人和答辩委员,
然后根据论文评阅人和答辩委员对论文的意见进行修改后,才能正式印刷,提交学校存档。

{\hei 十四、论文制作时不须页眉}
