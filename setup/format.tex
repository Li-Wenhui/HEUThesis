% !TEX TS-program = XeLaTeX
% !TEX encoding = UTF-8 Unicode

%%%%%%%%%%%%%%%%%%%%%%%%%%%%%%%%%%%%%%%%%%%%%%%%%%%%%%%%%%%%%%%%%%%%%%
%
%  哈尔滨工程大学学位论文 XeLaTeX 模版 —— 格式文件 format.tex
%
%  版本:1.0.0
%  最后更新:
%  修改者:Leo LiWenhui lwh@hrbeu.edu.cn
%  修订者:
%  编译环境1:Ubuntu 12.04 + TeXLive 2013/2014
%  编译环境2:Windows 7/8  + TeXLive 2013/2014
%
%%%%%%%%%%%%%%%%%%%%%%%%%%%%%%%%%%%%%%%%%%%%%%%%%%%%%%%%%%%%%%%%%%%%%%

%%%%%%%%%%%%%%%%%%%%%%%%%%%%%%%%%%%%%%%%%%%%%%%%%%%%%%%%%%%%%%%%%%%%%%
% 页面设置
%%%%%%%%%%%%%%%%%%%%%%%%%%%%%%%%%%%%%%%%%%%%%%%%%%%%%%%%%%%%%%%%%%%%%%
% A4 纸张
\setlength{\paperwidth}{210mm}
\setlength{\paperheight}{297mm}

% 设置正文尺寸大小
\setlength{\textwidth}{160mm}
\setlength{\textheight}{241mm}

% 设置正文区在正中间
\newlength \mymargin
\setlength{\mymargin}{(\paperwidth-\textwidth)/2}
\setlength{\oddsidemargin}{(\mymargin)-1in}
\setlength{\evensidemargin}{(\mymargin)-1in}

% 设置正文区偏移量,奇数页向右偏,偶数页向左偏
\newlength \myshift
\setlength{\myshift}{0mm}    % 双面打印的奇偶页偏移值,可根据需要修改,建议小于 5mm
\addtolength{\oddsidemargin}{\myshift}
\addtolength{\evensidemargin}{-\myshift}

% 页眉页脚相关距离设置
\setlength{\voffset}{-5.4mm} % 设置水平基线位置
\setlength{\topmargin}{0mm}  % 设置页眉距水平基线位置
\setlength{\headheight}{5mm} % 设置页眉高度
\setlength{\headsep}{8mm}    % 页眉与正文的距离
\setlength{\footskip}{8mm}   % 页脚与正文距离

% 公式的精调
\allowdisplaybreaks[4]  % 可以让公式在排不下的时候分页排,这可避免页面有大段空白。

% 下面这组命令使浮动对象的缺省值稍微宽松一点,从而防止幅度
% 对象占据过多的文本页面,也可以防止在很大空白的浮动页上放置很小的图形。
\renewcommand{\topfraction}{0.9999999}
\renewcommand{\textfraction}{0.0000001}
\renewcommand{\floatpagefraction}{0.9999}

% defaultfont 默认字体命令
\def\defaultfont{\renewcommand{\baselinestretch}{1.25} \daxiaosi}
%  \fontsize{12pt}{15pt}\selectfont}

% 设置目录字体和行间距
\def\defaultmenufont{\renewcommand{\baselinestretch}{1.22} \xiaosi}
%  \fontsize{12pt}{15pt}\selectfont}

% 固定距离内容填入及下划线
\makeatletter
\newcommand\fixeddistanceleft[2][10mm]{{\hb@xt@ #1{#2\hss}}}
\newcommand\fixeddistancecenter[2][10mm]{{\hb@xt@ #1{\hss#2\hss}}}
\newcommand\fixeddistanceright[2][10mm]{{\hb@xt@ #1{\hss#2}}}
\newcommand\fixedunderlineleft[2][10mm]{\underline{\hb@xt@ #1{#2\hss}}}
\newcommand\fixedunderlinecenter[2][10mm]{\underline{\hb@xt@ #1{\hss#2\hss}}}
\newcommand\fixedunderlineright[2][10mm]{\underline{\hb@xt@ #1{\hss#2}}}
\makeatother

%%%%%%%%%%%%%%%%%%%%%%%%%%%%%%%%%%%%%%%%%%%%%%%%%%%%%%%%%%%%%%%%%%%%%%
% 标题环境相关
%%%%%%%%%%%%%%%%%%%%%%%%%%%%%%%%%%%%%%%%%%%%%%%%%%%%%%%%%%%%%%%%%%%%%%
%判断单双面打印类型
\newif\ifoneortwosidetwoside
\newif\ifoneortwosideoneside

\def\temp{twoside}
\ifx\temp\oneortwoside
  \oneortwosidetwosidetrue  \oneortwosideonesidefalse
\fi

\def\temp{oneside}
\ifx\temp\oneortwoside
  \oneortwosidetwosidefalse  \oneortwosideonesidetrue
\fi

%判断论文类型
% 声明三个论文类型逻辑型变量
\newif\ifxueweidoctor
\newif\ifxueweimaster
\newif\ifxueweibachelor

% 根据 \xuewei 的定义为 \xueweidoctor \xueweimaster \xueweibachelor 赋值
\def\temp{Doctor}
\ifx\temp\xuewei % \ifx 用于判断两个变量是否匹配
  \xueweidoctortrue  \xueweimasterfalse \xueweibachelorfalse
\fi
\def\temp{Master}
\ifx\temp\xuewei
  \xueweidoctorfalse  \xueweimastertrue \xueweibachelorfalse
\fi
\def\temp{Bachelor}
\ifx\temp\xuewei
  \xueweidoctorfalse  \xueweimasterfalse \xueweibachelortrue
\fi

\ifxueweidoctor
  \newcommand{\cnxuewei}{博士}
  \newcommand{\enxuewei}{Doctor}
\fi

\ifxueweimaster
  \newcommand{\cnxuewei}{硕士}
  \newcommand{\enxuewei}{Master}
\fi

\ifxueweibachelor
  \newcommand{\cnxuewei}{学士}
  \newcommand{\enxuewei}{Bachelor}
\fi

%定义 学科 学位
\def \xuekeEngineering {Engineering}
\def \xuekeScience     {Science}
\def \xuekeManagement  {Management}
\def \xuekeArts        {Arts}
\def \xuekePhilosophy  {Philosophy}
\def \xuekeEconomics   {Economics}
\def \xuekeLaws        {Laws}
\def \xuekeEducation   {Education}
\def \xuekeHistory     {History}


\ifx \xueke \xuekeEngineering
\newcommand{\cnxueke}{工学}
\newcommand{\enxueke}{Engineering}
\newcommand{\enxk}   {Eng}
\fi

\ifx \xueke \xuekeScience
\newcommand{\cnxueke}{理学}
\newcommand{\enxueke}{Science}
\newcommand{\enxk}   {Sci}
\fi

\ifx \xueke \xuekeManagement
\newcommand{\cnxueke}{管理学}
\newcommand{\enxueke}{Management}
\newcommand{\enxk}   {Man}
\fi

\ifx \xueke \xuekeArts
\newcommand{\cnxueke}{文学}
\newcommand{\enxueke}{Arts}
\newcommand{\enxk}   {Art}
\fi

\ifx \xueke \xuekePhilosophy
\newcommand{\cnxueke}{哲学}
\newcommand{\enxueke}{Philosophy}
\newcommand{\enxk}   {Phi}
\fi

\ifx \xueke \xuekeEconomics
\newcommand{\cnxueke}{经济学}
\newcommand{\enxueke}{Economics}
\newcommand{\enxk}   {Eco}
\fi

\ifx \xueke \xuekeLaws
\newcommand{\cnxueke}{法学}
\newcommand{\enxueke}{Laws}
\newcommand{\enxk}   {Law}
\fi

\ifx \xueke \xuekeEducation
\newcommand{\cnxueke}{教育学}
\newcommand{\enxueke}{Education}
\newcommand{\enxk}   {Edu}
\fi

\ifx \xueke \xuekeHistory
\newcommand{\cnxueke}{历史学}
\newcommand{\enxueke}{History}
\newcommand{\enxk}   {His}
\fi

% 定义、定理等环境
\theoremstyle{plain}
\theoremheaderfont{\hei\bf}
\theorembodyfont{\song\rmfamily}
\newtheorem{definition}{\hei 定义}[chapter]
\newtheorem{example}{\hei 例}[chapter]
\newtheorem{algorithm}{\hei 算法}[chapter]
\newtheorem{theorem}{\hei 定理}[chapter]
\newtheorem{axiom}{\hei 公理}[chapter]
\newtheorem{proposition}[theorem]{\hei 命题}
\newtheorem{property}{\hei 性质}
\newtheorem{lemma}[theorem]{\hei 引理}
\newtheorem{corollary}{\hei 推论}[chapter]
\newtheorem{remark}{\hei 注解}[chapter]
\newenvironment{proof}{\hei{证明} }{\hfill $\square$ \vskip 4mm}

% 目录标题
\renewcommand{\contentsname}{\hfill \hei 目~~~~录 \hfill}
\renewcommand{\listfigurename}{\hfill 插~图~目~录 \hfill}
\renewcommand{\listtablename}{\hfill 表~格~目~录 \hfill}
\renewcommand{\bibname}{\hfill 参~考~文~献 \hfill}
\renewcommand\appendixname{附~录}

%%%%%%%%%%%%%%%%%%%%%%%%%%%%%%%%%%%%%%%%%%%%%%%%%%%%%%%%%%%%%%%%%%%%%%
% 段落章节相关
%%%%%%%%%%%%%%%%%%%%%%%%%%%%%%%%%%%%%%%%%%%%%%%%%%%%%%%%%%%%%%%%%%%%%%
\setcounter{secnumdepth}{4}
\setcounter{tocdepth}{4}
\setcounter{chapter}{0}

% 设置章、节、小节、小小节的间距
%\titleformat{\chapter}[hang]{\normalfont\xiaosan\hei\sf}{\xiaosan\thechapter}{1em}{\xiaosan}
\titleformat{\chapter}{\centering\xiaoer\hei\sf}{第\,\thechapter\,章}{1em}{} % 设置中文章格式中央对齐:第  章
\titlespacing{\chapter}{0pt}{-3ex  plus .1ex minus .2ex}{3.3ex}
\titleformat{\section}[hang]{\xiaosan\hei\sf}{\xiaosan\thesection}{1em}{}{}
\titlespacing{\section}{0pt}{0.5em}{0.5em}
\titleformat{\subsection}[hang]{\sihao\hei\sf}{\sihao\thesubsection}{1em}{}{}
\titlespacing{\subsection}{0pt}{0.5em}{0.3em}
\titleformat{\subsubsection}[hang]{\xiaosi\hei\sf}{\xiaosi\thesubsubsection}{1em}{}{}
\titlespacing{\subsubsection}{0pt}{0.3em}{0pt}

% 缩小目录中各级标题之间的缩进
% \dottedcontents{<section>}[<left>]{<above>}{<labelwidth>}{<leaderwidth>}
\dottedcontents{chapter}[3mm]{\vspace{0.2em}}{1.0em}{5pt}
\dottedcontents{section}[13mm]{}{1.8em}{5pt}
\dottedcontents{subsection}[23mm]{}{2.7em}{5pt}
\dottedcontents{subsubsection}[33mm]{}{3.4em}{5pt}

% 设置目录中各级标题之间的缩进
\makeatletter
\renewcommand*{\l@chapter}{\@dottedtocline{0}{0em}{5em}}% 细点\@dottedtocline  粗点\@dottedtoclinebold
\renewcommand*{\l@section}{\@dottedtocline{1}{1em}{1.8em}}
\renewcommand*{\l@subsection}{\@dottedtocline{2}{2em}{2.5em}}
\renewcommand*{\l@subsubsection}{\@dottedtocline{3}{2em}{2.5em}}

% 设置章标题格式
% \titlecontents{章节名称}[左端距离]{标题字体、与上文间距等}{标题序号}{空}{引导符和页码}[与下文间距]
\titlecontents{chapter}[3.8em]{\hspace{-3.8em}\hei}{第~\thecontentslabel~章~~}{}{\titlerule*[4pt]{.}\contentspage}

% 段落之间的竖直距离
\setlength{\parskip}{1.2pt}
% 段落缩进
\setlength{\parindent}{24pt}
% 定义行距
\renewcommand{\baselinestretch}{1.25}
% 参考文献条目间行间距
\setlength{\bibsep}{2pt}


%%%%%%%%%%%%%%%%%%%%%%%%%%%%%%%%%%%%%%%%%%%%%%%%%%%%%%%%%%%%%%%%%%%%%%
% 页眉页脚设置
%%%%%%%%%%%%%%%%%%%%%%%%%%%%%%%%%%%%%%%%%%%%%%%%%%%%%%%%%%%%%%%%%%%%%%

\newcommand{\makeheadrule}{%
    \rule[12pt]{\textwidth}{0.5pt} \\[-23pt]
    \rule{\textwidth}{2.0pt}
  \vskip-.8\baselineskip}

\makeatletter
\renewcommand{\headrule}{%
  {\if@fancyplain\let\headrulewidth\plainheadrulewidth\fi
    \makeheadrule}}

\pagestyle{fancyplain}

%去掉章节标题中的数字
%%不要注销这一行,否则页眉会变成:“第1章1  绪论”样式
\renewcommand{\chaptermark}[1]{\markboth{第\thechapter 章~~~~\ #1}{}}
\fancyhf{}

% 附录设置:附录不编章节号,但列入目录和页眉
\renewcommand{\appendix}[1]{%
    \chapter*{#1}%
    \addcontentsline{toc}{chapter}{#1}%
    \markboth{#1}{#1}
}

%在book文件类别下,\leftmark自动存录各章之章名,\rightmark记录节标题
%根据单双面打印设置不同的页眉;
\fancyhead[CO]{\song\xiaosi{\leftmark}}
\fancyhead[CE]{\song\xiaosi{\@cnuniversty\cnxuewei 学位论文}}
\fancyfoot[C,C]{\xiaosi$-$~\thepage~$-$}

%偶数页为空白页面时处理
\makeatletter
\def\cleardoublepage{\clearpage\if@twoside \ifodd\c@page\else%
  \hbox{}%
%  \thispagestyle{empty}%   % 清除页眉、页脚
  \vspace*{80mm}
  \centerline{\xiaoer\song {{(此页无正文)}}}
  \centerline{\xiaosi\song {{(THIS PAGE IS INTENTIONALLY LEFT BLANK)}}}
  \newpage%
  \if@twocolumn\hbox{}\newpage\fi\fi\fi}

%%%%%%%%%%%%%%%%%%%%%%%%%%%%%%%%%%%%%%%%%%%%%%%%%%%%%%%%%%%%%%%%%%%%%%
% 列表环境设置
%%%%%%%%%%%%%%%%%%%%%%%%%%%%%%%%%%%%%%%%%%%%%%%%%%%%%%%%%%%%%%%%%%%%%%

\setlist[enumerate]{(1),itemsep=-5pt,topsep=0mm,labelindent=\parindent,leftmargin=*}

%%%%%%%%%%%%%%%%%%%%%%%%%%%%%%%%%%%%%%%%%%%%%%%%%%%%%%%%%%%%%%%%%%%%%%
% 国际单位,以点连接。
%%%%%%%%%%%%%%%%%%%%%%%%%%%%%%%%%%%%%%%%%%%%%%%%%%%%%%%%%%%%%%%%%%%%%%
\sisetup{inter-unit-product = { }\cdot{ }}

%%%%%%%%%%%%%%%%%%%%%%%%%%%%%%%%%%%%%%%%%%%%%%%%%%%%%%%%%%%%%%%%%%%%%%
% 参考文献的处理
%%%%%%%%%%%%%%%%%%%%%%%%%%%%%%%%%%%%%%%%%%%%%%%%%%%%%%%%%%%%%%%%%%%%%%

% \addtolength{\bibsep}{-0.5 em}      % 缩小参考文献间的垂直间距
% 上标引用,比\cite位置更偏上、字号稍小
\DeclareRobustCommand\scite{\@scite}
\def\@scite#1{\textsuperscript{\cite{#1}}}
% 行间引用,与正文格式一致
\DeclareRobustCommand\lcite{\@lcite}
\def\@lcite#1{\begingroup\let\@cite\NAT@citenum\citep{#1}\endgroup}

\setlength{\bibhang}{2em}
\bibpunct{[}{]}{,}{s}{}{}


%%%%%%%%%%%%%%%%%%%%%%%%%%%%%%%%%%%%%%%%%%%%%%%%%%%%%%%%%%%%%%%%%%%%%%
%   其他设置
%%%%%%%%%%%%%%%%%%%%%%%%%%%%%%%%%%%%%%%%%%%%%%%%%%%%%%%%%%%%%%%%%%%%%%
% 使图编号为 7-1 的格式 %\protect{~}
\renewcommand{\thefigure}{\arabic{chapter}-\arabic{figure}}
% 使子图编号为 (a)的格式
\renewcommand{\thesubfigure}{(\alph{subfigure})}
% 使子图引用为 7-1 (a) 的格式,母图编号和子图编号之间用~加一个空格
\renewcommand{\p@subfigure}{\thefigure~}
% 使表编号为 7-1 的格式
\renewcommand{\thetable}{\arabic{chapter}-\arabic{table}}
% 使公式编号为 7-1 的格式
\renewcommand{\theequation}{\arabic{chapter}-\arabic{equation}}

%插图索引格式: 图 x. 图标题 ......页码
\renewcommand\listoffigures{%
    \chapter*{\listfigurename}%
    \addcontentsline{toc}{chapter}{插图目录}
    \markboth{\listfigurename}{\listfigurename}
    \renewcommand{\numberline}[1]{图~##1~~}
    \@starttoc{lof}%
    }

%表格索引格式: 表 x. 表标题 ......页码
\renewcommand\listoftables{%
    \chapter*{\listtablename}%
    \addcontentsline{toc}{chapter}{表格目录}
    \markboth{\listtablename}{\listtablename}
    \renewcommand{\numberline}[1]{表~##1~~}
    \@starttoc{lot}%
    }

%%%%%%%%%%%%%%%%%%%%%%%%%%%%%%%%%%%%%%%%%%%%%%%%%%%%%%%%%%%%%%%%%%%%%%
%   图形表格
%%%%%%%%%%%%%%%%%%%%%%%%%%%%%%%%%%%%%%%%%%%%%%%%%%%%%%%%%%%%%%%%%%%%%%
\renewcommand{\figurename}{图}
\renewcommand{\tablename}{表}
% \captionstyle{\centering}
% \hangcaption
\captiondelim{\hspace{1em}}
\captiondelim{\hspace{1em}}
\captionnamefont{\zhongwu}
\captiontitlefont{\zhongwu}
\setlength{\abovecaptionskip}{0pt}
\setlength{\belowcaptionskip}{0pt}

\newcommand{\tablepage}[2]{\begin{minipage}{#1}\vspace{0.5ex} #2 \vspace{0.5ex}\end{minipage}}
\newcommand{\returnpage}[2]{\begin{minipage}{#1}\vspace{0.5ex} #2 \vspace{-1.5ex}\end{minipage}}

%%%%%%%%%%%%%%%%%%%%%%%%%%%%%%%%%%%%%%%%%%%%%%%%%%%%%%%%%%%%%%%%%%%%%%
%   脚注格式设置
%%%%%%%%%%%%%%%%%%%%%%%%%%%%%%%%%%%%%%%%%%%%%%%%%%%%%%%%%%%%%%%%%%%%%%
  %使用pifont包里面ding产生带圈的数字1~10
\newcommand\chnnocirc[1]{%
\ifcase#1 a \or {\ding{172}} \or {\ding{173}} \or {\ding{174}} \or {\ding{175}} \or {\ding{176}} \or {\ding{177}} \or {\ding{178}} \or {\ding{179}} \or {\ding{180}}\fi}
\renewcommand{\thefootnote}{\chnnocirc{\arabic{footnote}}}

% 自定义一个空命令,用于注释掉文本中不需要的部分。
\newcommand{\comment}[1]{}

% 双语章节重新定义BiChapter、BiSection等命令,可实现标题手动换行,但不影响目录
\def\BiChapter{\relax\@ifnextchar [{\@BiChapter}{\@@BiChapter}}
\def\@BiChapter[#1]#2#3{\chapter[#1]{#2}
    \addcontentsline{toe}{chapter}{\bfseries \xiaosi Chapter \thechapter\hspace{0.5em} #3}}
\def\@@BiChapter#1#2{\chapter{#1}
    \addcontentsline{toe}{chapter}{\bfseries \xiaosi Chapter \thechapter\hspace{0.5em}{\boldmath #2}}}

\newcommand{\BiSection}[2]
{   \section{#1}
    \addcontentsline{toe}{section}{\protect\numberline{\csname thesection\endcsname}#2}
}

\newcommand{\BiSubsection}[2]
{    \subsection{#1}
    \addcontentsline{toe}{subsection}{\protect\numberline{\csname thesubsection\endcsname}#2}
}

\newcommand{\BiSubsubsection}[2]
{    \subsubsection{#1}
    \addcontentsline{toe}{subsubsection}{\protect\numberline{\csname thesubsubsection\endcsname}#2}
}

\newcommand{\BiAppendix}[2] % 该附录命令适用于发表文章,简历等
{\phantomsection
\markboth{#1}{#1}
\addcontentsline{toc}{chapter}{\xiaosi #1}
\addcontentsline{toe}{chapter}{\bfseries \xiaosi #2}  \chapter*{#1}
}

\newcommand{\BiAppChapter}[2]    % 该附录命令适用于有章节的完整附录
{\phantomsection
 \chapter{#1}
 \addcontentsline{toe}{chapter}{\bfseries \xiaosi Appendix \thechapter~~#2}
}

\def\engcontentsname{\uppercase{CONTENTS}}
\newcommand\tableofengcontents{
   \pdfbookmark[0]{\uppercase{CONTENTS}}{encontent}
   \chapter*{\engcontentsname  %chapter*上移一行,避免在toc中出现
       \@mkboth{%
          \engcontentsname}{\engcontentsname}}
   \@starttoc{toe}%
}


%%%%%%%%%%%%%%%%%%%%%%%%%%%%%%%%%%%%%%%%%%%%%%%%%%%%%%%%%%%%%%%%%%%%%%%%%%%%%%%%
% 封面摘要
%%%%%%%%%%%%%%%%%%%%%%%%%%%%%%%%%%%%%%%%%%%%%%%%%%%%%%%%%%%%%%%%%%%%%%%%%%%%%%%%
\def\cnauthorno#1{\def\@cnauthorno{#1}}\def\@cnauthorno{}

\def\cntitle#1{\def\@cntitle{#1}}\def\@cntitle{}
\def\cnaffil#1{\def\@cnaffil{#1}}\def\@cnaffil{}
\def\cnsubject#1{\def\@cnsubject{#1}}\def\@cnsubject{}
\def\cnauthor#1{\def\@cnauthor{#1}}\def\@cnauthor{}
\def\cnsupervisor#1{\def\@cnsupervisor{#1}}\def\@cnsupervisor{}
\def\cnsupervisortitle#1{\def\@cnsupervisortitle{#1}}\def\@cnsupervisortitle{}
\def\cnsubdate#1{\def\@cnsubdate{#1}}\def\@cnsubdate{}
\def\cndefdate#1{\def\@cndefdate{#1}}\def\@cndefdate{}
\long\def\cnabstract#1{\long\def\@cnabstract{#1}}\long\def\@cnabstract{}
\def\cnkeywords#1{\def\@cnkeywords{#1}}\def\@cnkeywords{}
\def\cnreviewer#1{\def\@cnreviewer{#1}}\def\@cnreviewer{}

\def\entitle#1{\def\@entitle{#1}}\def\@entitle{}
\def\enaffil#1{\def\@enaffil{#1}}\def\@enaffil{}
\def\ensubject#1{\def\@ensubject{#1}}\def\@ensubject{}
\def\enauthor#1{\def\@enauthor{#1}}\def\@enauthor{}
\def\ensupervisor#1{\def\@ensupervisor{#1}}\def\@ensupervisor{}
\def\ensupervisortitle#1{\def\@ensupervisortitle{#1}}\def\@ensupervisortitle{}
\def\ensubdate#1{\def\@ensubdate{#1}}\def\@ensubdate{}
\def\endefdate#1{\def\@endefdate{#1}}\def\@endefdate{}
\long\def\enabstract#1{\long\def\@enabstract{#1}}\long\def\@enabstract{}
\def\enkeywords#1{\def\@enkeywords{#1}}\def\@enkeywords{}
\def\enreviewer#1{\def\@enreviewer{#1}}\def\@enreviewer{}

\long\def\NotationList#1{\long\def\@NotationList{#1}}\long\def\@NotationList{}
\long\def\cnauthorization#1{\long\def\@cnauthorization{#1}}\long\def\@cnauthorization{}

\def\natclassifiedindex#1{\def\@natclassifiedindex{#1}}\def\@natclassifiedindex{}
\def\internatclassifiedindex#1{\def\@internatclassifiedindex{#1}}\def\@internatclassifiedindex{}

\def\studentno#1{\def\@studentno{#1}}\def\@studentno{}

\def\cnstatesecrets#1{\def\@cnstatesecrets{#1}}\def\@cnstatesecrets{}
\def\enstatesecrets#1{\def\@enstatesecrets{#1}}\def\@enstatesecrets{}

\def\cnuniversty#1{\def\@cnuniversty{#1}}\def\@cnuniversty{}
\def\enuniversty#1{\def\@euniversity{#1}}\def\@euniversity{}

% 封面
\def\makecover{
  \begin{titlepage}
    \newpage
    \thispagestyle{empty}
    \begin{center}
    \ifxueweidoctor
            \vspace*{5mm}
            \renewcommand{\arraystretch}{1.5}
            \song \xiaosi{
                \begin{tabular}{@{}r@{:}l@{}}
                    分类号               & \underline{\makebox[6em][c]{\@natclassifiedindex}} \\
                    U \hfill D \hfill  C & \underline{\makebox[6em][c]{\@internatclassifiedindex}}
                \end{tabular}}\hfill
            \song \xiaosi{
                \begin{tabular}{@{}r@{:}l@{}}
                    密 \ 级 & \underline{\makebox[6em][c]{\@cnstatesecrets}} \\
                    编 \ 号 & \underline{\makebox[6em][c]{\@studentno}}
                \end{tabular}}
            \renewcommand{\arraystretch}{1}

            \vspace*{20mm}
            \centerline{\xiaoer\song {\cnxueke\cnxuewei{学位论文}}}
            \vspace{5mm}

            \parbox[t][30mm][t]{\textwidth}{
            \begin{center}\erhao\hei{\@cntitle}\end{center} }

            \vspace*{30mm}
             \parbox[t][40mm][b]{\textwidth}
              {\xiaosan
             \begin{center} \renewcommand{\arraystretch}{1.25} \song
                 \begin{tabular}{l@{:}l}
                 {\xiaosan 博 \hfill 士 \hfill 研\hfill 究\hfill 生}  & \@cnauthor      \ \\
                 {\xiaosan 指 \hfill 导 \hfill 教 \hfill 师}          & \@cnsupervisor  \\
                 {\xiaosan 学 \hfill 科 \hfill 专 \hfill 业}          & \@cnsubject \\
                 {\xiaosan 学位论文主审人}                            & \@cnreviewer
                 \end{tabular} \renewcommand{\arraystretch}{1}
             \end{center} }
        \fi

        \ifxueweimaster
            \vspace*{5mm}
            \renewcommand{\arraystretch}{1.5}
            \song \xiaosi{
                \begin{tabular}{@{}r@{:}l@{}}
                    分类号               & \underline{\makebox[6em][c]{\@natclassifiedindex}} \\
                    U \hfill D \hfill  C & \underline{\makebox[6em][c]{\@internatclassifiedindex}}
                \end{tabular}}\hfill
            \song \xiaosi{
                \begin{tabular}{@{}r@{:}l@{}}
                    密 \ 级 & \underline{\makebox[6em][c]{\@cnstatesecrets}} \\
                    编 \ 号 & \underline{\makebox[6em][c]{\@studentno}}
                \end{tabular}}
            \renewcommand{\arraystretch}{1}

            \vspace*{20mm}
            \centerline{\xiaoer\song {\cnxueke\cnxuewei{学位论文}}}
            \vspace{5mm}

            \parbox[t][30mm][t]{\textwidth}{
            \begin{center}\erhao\hei{\@cntitle}\end{center} }

            \vspace*{30mm}
             \parbox[t][40mm][b]{\textwidth}
              {\xiaosan
             \begin{center} \renewcommand{\arraystretch}{1.25} \song
                 \begin{tabular}{l@{:}l}
                 {\xiaosan 硕 \hfill 士 \hfill 研\hfill 究\hfill 生}  & \@cnauthor      \\
                 {\xiaosan 指 \hfill 导 \hfill 教 \hfill 师}          & \@cnsupervisor  \\
                 {\xiaosan 学 \hfill 科 \hfill 专 \hfill 业}          & \@cnsubject \\
                 {\xiaosan 学位论文主审人}                            & \@cnreviewer
                 \end{tabular} \renewcommand{\arraystretch}{1}
             \end{center} }
        \fi

        \ifxueweibachelor
            \vspace*{5mm}
            \renewcommand{\arraystretch}{1.5}
            {\song \xiaosi
                \begin{tabular}{@{}r@{:}l@{}}
                    ~ & ~ \\
                    ~ & ~
                \end{tabular}}\hfill
            {\song \xiaosi
                \begin{tabular}{@{}r@{:}l@{}}
                    密 \ 级 & \underline{\makebox[6em][c]{\@cnstatesecrets}} \\
                    学 \ 号 & \underline{\makebox[6em][c]{\@studentno}}
                \end{tabular}}
            \renewcommand{\arraystretch}{1}

            \vspace*{20mm}
            \centerline{\xiaoer\song {\@cnuniversty\cnxuewei{学位论文}}}
            \vspace{5mm}

            \parbox[t][30mm][t]{\textwidth}{
            \begin{center}\erhao\hei{\@cntitle}\end{center} }

             \vspace*{30mm}
             \parbox[t][40mm][b]{\textwidth}
              {\xiaosan
             \begin{center} \renewcommand{\arraystretch}{1.25} \song
                 \begin{tabular}{l@{:}l}
                 {\xiaosan 院(系)名称院}                 & \@cnaffil  \\
                 {\xiaosan 专\hfill 业\hfill 名\hfill 称}  & \@cnsubject \\
                 {\xiaosan 学\hfill 生\hfill 姓\hfill 名}  & \@cnauthor \\
                 {\xiaosan 指\hfill 导\hfill 教\hfill 师}  & \@cnsupervisor
                 \end{tabular} \renewcommand{\arraystretch}{1}
             \end{center} }
        \fi

      \vspace{30mm}

      {
        \xiaoer\kai {\@cnuniversty} \\
        \vspace{4mm}
        \xiaosan\song {\@cnsubdate}
      }
    \end{center}

    % 双面打印时封面后加空白页
    \ifoneortwosidetwoside
      \newpage
      ~~~\vspace{1em}
      \thispagestyle{empty}
    \fi

% % % % % % % % % % % % % % % % % % % % % % % % % % % % % % % % % % % % % % % % % % % % % % % % % % %
    %内封(扉页)
    \newpage
    \thispagestyle{empty}
    \pdfbookmark[0]{\@cntitle}{cntitlepage}
    \begin{center}
        \ifxueweidoctor
            \vspace*{5mm}
            \begin{center}
                \renewcommand{\arraystretch}{1.5}
                {\song \xiaosi
                    \begin{tabular}{@{}r@{:}l@{}}
                        分类号              & \underline{\makebox[6em][c]{\@natclassifiedindex}} \\
                        U \hfill D \hfill C & \underline{\makebox[6em][c]{\@internatclassifiedindex}}
                    \end{tabular}}\hfill
                {\song \xiaosi
                    \begin{tabular}{@{}r@{:}l@{}}
                        密 \ 级 & \underline{\makebox[6em][c]{\@cnstatesecrets}} \\
                        编 \ 号 & \underline{\makebox[6em][c]{\@studentno}}
                    \end{tabular}}
                \renewcommand{\arraystretch}{1}

                \vspace*{20mm}

                \centerline{\song\xiaoer{\cnxueke\cnxuewei{学位论文}}}
                \vspace{5mm}
                \parbox[t][30mm][t]{\textwidth}{
                \begin{center}\erhao\hei{\@cntitle}\end{center}}

                \vspace*{20mm}

                \parbox[t][80mm][b]{\textwidth}
                 {\sihao
                \begin{center} \renewcommand{\arraystretch}{1.25} \song
                \begin{tabular}{l@{:}l}
                    \textbf{\sihao 博\hfill 士\hfill 研\hfill 究\hfill 生} & \@cnauthor  \\
                    \textbf{\sihao 指\hfill 导\hfill 教\hfill 师}          & \@cnsupervisor  \\
                    \textbf{\sihao 学\hfill 位\hfill 级\hfill 别}          & \cnxueke\cnxuewei  \\
                    \textbf{\sihao 学\hfill 科\hfill 专\hfill 业}          & \@cnsubject  \\
                    \textbf{\sihao 所\hfill 在\hfill 单\hfill 位}          & \@cnaffil  \\
                    \textbf{\sihao 论文提交日期}                           & \@cnsubdate  \\
                    \textbf{\sihao 论文答辩日期}                           & \@cndefdate  \\
                    \textbf{\sihao 学位授予单位}                           & {\@cnuniversty}
                \end{tabular} \renewcommand{\arraystretch}{1}
                \end{center}}
            \end{center}
        \fi

        \ifxueweimaster
            \vspace*{5mm}
            \begin{center}
                \renewcommand{\arraystretch}{1.5}
                {\song \xiaosi
                    \begin{tabular}{@{}r@{:}l@{}}
                        分类号              & \underline{\makebox[6em][c]{\@natclassifiedindex}} \\
                        U \hfill D \hfill C & \underline{\makebox[6em][c]{\@internatclassifiedindex}}
                    \end{tabular}}\hfill
                {\song \xiaosi
                    \begin{tabular}{@{}r@{:}l@{}}
                        密 \ 级 & \underline{\makebox[6em][c]{\@cnstatesecrets}} \\
                        编 \ 号 & \underline{\makebox[6em][c]{\@studentno}}
                    \end{tabular}}
                \renewcommand{\arraystretch}{1}

                \vspace*{20mm}
                \centerline{\xiaoer\song {\cnxueke\cnxuewei{学位论文}}}
                \vspace*{5mm}
                \parbox[t][30mm][t]{\textwidth}{
                \begin{center}\erhao\hei{\@cntitle}\end{center}}

                \vspace*{20mm}

                \parbox[t][80mm][b]{\textwidth}
                 {\sihao
                \begin{center} \renewcommand{\arraystretch}{1.25} \song
                \begin{tabular}{l@{:}l}
                    \textbf{\sihao 硕\hfill 士\hfill 研\hfill 究\hfill 生} & \@cnauthor  \\
                    \textbf{\sihao 指\hfill 导\hfill 教\hfill 师}          & \@cnsupervisor  \\
                    \textbf{\sihao 学\hfill 位\hfill 级\hfill 别}          & \cnxueke\cnxuewei  \\
                    \textbf{\sihao 学\hfill 科\hfill 专\hfill 业}          & \@cnsubject  \\
                    \textbf{\sihao 所\hfill 在\hfill 单\hfill 位}          & \@cnaffil  \\
                    \textbf{\sihao 论文提交日期}                           & \@cnsubdate  \\
                    \textbf{\sihao 论文答辩日期}                           & \@cndefdate  \\
                    \textbf{\sihao 学位授予单位}                           & {\@cnuniversty}
                \end{tabular} \renewcommand{\arraystretch}{1}
                \end{center}}
            \end{center}
        \fi

        \ifxueweibachelor
            \vspace*{5mm}
            \begin{center}
                \renewcommand{\arraystretch}{1.5}
                {\song \xiaosi
                    \begin{tabular}{@{}r@{:}l@{}}
                        ~ & ~ \\
                        ~ & ~
                    \end{tabular}}\hfill
                {\song \xiaosi
                    \begin{tabular}{@{}r@{:}l@{}}
                        密 \ 级 & \underline{\makebox[6em][c]{\@cnstatesecrets}} \\
                        学 \ 号 & \underline{\makebox[6em][c]{\@studentno}}
                    \end{tabular}}
                \renewcommand{\arraystretch}{1}

                \vspace*{20mm}

                \parbox[t][30mm][t]{\textwidth}{
                \begin{center}\erhao\hei{\@cntitle}\end{center}}
                \vspace*{5mm}
                \parbox[t][30mm][t]{\textwidth}{
                \begin{center}\erhao\hei{\@entitle}\end{center}}

                \vspace*{20mm}

                \parbox[t][80mm][b]{\textwidth}
                 {\sihao
                \begin{center} \renewcommand{\arraystretch}{1.25} \song
                \begin{tabular}{l@{:}l}
                    \textbf{\sihao 学\hfill 生\hfill 姓\hfill 名}  & \@cnauthor  \\
                    \textbf{\sihao 所\hfill 在\hfill 学\hfill 院}  & \@cnaffil  \\
                    \textbf{\sihao 所\hfill 在\hfill 专\hfill 业}  & \@cnsubject  \\
                    \textbf{\sihao 指\hfill 导\hfill 教\hfill 师}  & \@cnsupervisor  \\
                    \textbf{\sihao 职\hfill                   称}  & \@cnsupervisortitle \\
                    \textbf{\sihao 所\hfill 在\hfill 单\hfill 位}  & \@cnaffil  \\
                    \textbf{\sihao 论文提交日期}                   & \@cnsubdate  \\
                    \textbf{\sihao 论文答辩日期}                   & \@cndefdate  \\
                    \textbf{\sihao 学位授予单位}                   & {\@cnuniversty}
                \end{tabular} \renewcommand{\arraystretch}{1}
                \end{center}}
            \end{center}
        \fi
      \end{center}

    % 双面打印时封面后加空白页
    \ifoneortwosidetwoside
      \newpage
      ~~~\vspace{1em}
      \thispagestyle{empty}
    \fi

% % % % % % % % % % % % % % % % % % % % % % % % % % % % % % % % % % % % % % % % % % % % % % % % % %
    % 英文封面
     \begin{center}
        \ifxueweidoctor
            \newpage
            \thispagestyle{empty}
            \pdfbookmark[0]{\uppercase{\@entitle}}{entitlepage}
            \vspace*{5mm}
            \begin{center}
                \renewcommand{\arraystretch}{1.5}
                {\song \xiaosi
                    \begin{tabular}{@{}r@{:}l@{}}
                        Classif\/ied Index & \@natclassifiedindex \\
                        U.D.C              & \@internatclassifiedindex
                    \end{tabular}}\hfill
                {\song \xiaosi
                    \begin{tabular}{@{}r@{ }l@{}}
                        ~ & ~ \\
                        ~ & ~
                    \end{tabular}}
                \renewcommand{\arraystretch}{1}

                \vspace*{30mm}

                \centerline{\song\xiaoer{A Dissertation for the Degree of D.\enxk }}
                \vspace*{5mm}
                \parbox[t][30mm][t]{\textwidth}{\erhao
                \begin{center} {\bfseries \@entitle}\end{center}}

                \vspace*{20mm}

                \parbox[t][80mm][t]{\textwidth}
                {\sihao\renewcommand{\arraystretch}{1.25}
                \begin{tabular}{r@{\textbf ~:~ }l@{~}}
                \textbf{Candidate}                     &  \@enauthor  \\
                \textbf{Supervisor}                    &  \@ensupervisor\\
                \textbf{Academic Degree Applied for}   &  \enxuewei~of~\enxueke \\
                \textbf{Specialty}                     &  \@ensubject  \\
                \textbf{Affiliation}                   &  \@enaffil  \\
                \textbf{Date of Defence}               &  \@endefdate  \\
                \textbf{Degree-Conferring-Institution} &  \@euniversity
                \end{tabular}\renewcommand{\arraystretch}{1}
                }
            \end{center}
            % 双面打印时封面后加空白页
            \ifoneortwosidetwoside
              \newpage
              ~~~\vspace{1em}
              \thispagestyle{empty}
            \fi
        \fi

        \ifxueweimaster
            \newpage
            \thispagestyle{empty}
            \pdfbookmark[0]{\uppercase{\@entitle}}{entitlepage}
            \vspace*{5mm}
            \begin{center}
                \renewcommand{\arraystretch}{1.5}
                {\song \xiaosi
                    \begin{tabular}{@{}r@{ }l@{}}
                        ~ & ~ \\
                        ~ & ~
                    \end{tabular}}\hfill
                {\song \xiaosi
                    \begin{tabular}{@{}r@{:}l@{}}
                        Classif\/ied Index & \@natclassifiedindex \\
                        U.D.C              & \@internatclassifiedindex
                    \end{tabular}}
                \renewcommand{\arraystretch}{1}

                \vspace*{20mm}

                \centerline{\song\xiaoer{A Dissertation for the Degree of M.\enxk }}
                \vspace*{5mm}
                \parbox[t][30mm][t]{\textwidth}{\erhao
                \begin{center} {\bfseries \@entitle}\end{center}}

                \vspace*{30mm}

                \parbox[t][80mm][t]{\textwidth}
                {\sihao\renewcommand{\arraystretch}{1.25}
                \begin{tabular}{r@{\textbf ~:~ }l@{~}}
                \textbf{Candidate}                     &  \@enauthor  \\
                \textbf{Supervisor}                    &  \@ensupervisor\\
                \textbf{Academic Degree Applied for}   &  \enxuewei~of~\enxueke \\
                \textbf{Specialty}                     &  \@ensubject  \\
                \textbf{Date of Submission}            &  \@ensubdate  \\
                \textbf{Date of Oral Examination}      &  \@endefdate  \\
                \textbf{University}                    &  \@euniversity
                \end{tabular}\renewcommand{\arraystretch}{1}
                }
            \end{center}

            % 双面打印时封面后加空白页
            \ifoneortwosidetwoside
              \newpage
              ~~~\vspace{1em}
              \thispagestyle{empty}
            \fi
        \fi

      \end{center}
    \cleardoublepage
  \end{titlepage}
}

% 原创性与使用授权说明
\def\authorization
{
    \thispagestyle{empty}
    \@cnauthorization
    \clearpage

    % 双面打印时加空白页
    \ifoneortwosidetwoside
      \newpage
      ~~~\vspace{1em}
      \thispagestyle{empty}
    \fi
}

\def\makeabstract{
  \defaultfont
  \chapter*{摘  要}
  \addcontentsline{toc}{chapter}{摘  要}
  \markboth{摘  要}{摘  要}
  \setcounter{page}{1}

  \@cnabstract
  \vspace{5mm}

  \noindent {\hei{关键词:{\fs\@cnkeywords}}}
  \defaultfont
  \cleardoublepage

  \chapter*{ABSTRACT}
  \addcontentsline{toc}{chapter}{ABSTRACT}
  \markboth{ABSTRACT}{ABSTRACT}

  \@enabstract
  \vspace{5mm}

  \noindent {\textbf{Key Words:}}~~{\textsf{\@enkeywords}}
  \defaultfont
  \cleardoublepage
}

\makeatletter
\def\hlinewd#1{%
  \noalign{\ifnum0=`}\fi\hrule \@height #1 \futurelet
  \reserved@a\@xhline}
\makeatother

% 定义索引生成
\def\generateindex
{
  \addcontentsline{toc}{chapter}{\indexname}
  \printindex
  \cleardoublepage
}

\raggedbottom 